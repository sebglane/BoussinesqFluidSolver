\section{Introduction\label{sec:Introduction}}
The objective of this code is to solve the following set of dimensionless equations for the velocity, $\bm{v}$, a pressure, $P$, the temperature, $T$, and the magnetic field, $\bm{B}$:
\begin{gather}
	\nabla\cdot\bm{v}=0\,,\quad
	\pd{\bm{v}}{t}+\bm{v}\cdot\nabla\bm{v}+C_1\bm{\Omega}\times\bm{v}=-\nabla P+C_2\laplace\bm{v}-C_3T\bm{g}+C_5(\nabla\times\bm{B})\times\bm{B}\,,\label{eqn:Momentum}\\
	\pd{T}{t}+\bm{v}\cdot\nabla T=C_4\laplace T\,,\label{eqn:Energy}\\
	\nabla\cdot\bm{B}=0\,,\qquad
	\pd{\bm{B}}{t}=\nabla\times(\bm{v}\times\bm{B})-C_5\nabla\times(\nabla\times\bm{B})\,,\label{eqn:MagneticInduction}
\end{gather}
The constants $C_1$ to $C_5$ are dimensionless and depend on the case studied. They are selected according to Table\,\ref{tbl:Dimensionless}. In case of a purely hydrodynamic problem the magnetic induction equation is not considered and the Lorentz force vanishes in the momentum equation.
\begin{table}[!htb]
	\centering
	\caption{Dimensionless numbers used in Eqs.\,\eqref{eqn:Momentum}-\eqref{eqn:MagneticInduction}.\label{tbl:Dimensionless}}
	\begin{tabular}{cccccc}
		\toprule
		case & $C_1$ & $C_2$ & $C_3$ & $C_4$ & $C_5$ \\\midrule
		non-rotating hydrodynamic & 0 & $\sqrt{\Prandtl/\Rayleigh}$ & 1 & $1/\sqrt{\Rayleigh\Prandtl}$ & 0\\
		rotating hydrodynamic & $2/\Ekman$ & 1 & $\Rayleigh/\Prandtl$ & $1/\Prandtl$ & 0\\
		rotating magnetohydrodynamic & $2/\Ekman$ & 1 & $\Rayleigh/\Prandtl$ & $1/\Prandtl$ & $1/\magPrandtl$\\
		\bottomrule
	\end{tabular}
\end{table}
The dimensionless numbers in Table\,\ref{tbl:Dimensionless} are defined as follows:
\begin{itemize}
	\item Rayleigh number: $\Rayleigh=\alpha g D^3/(\nu\kappa)$,
	\item Prandtl number: $\Prandtl=\nu/\kappa$,
	\item magnetic Prandtl number: $\magPrandtl=\nu/\eta$,
	\item Ekman number:	$\Ekman=\nu/(\Omega D^2)$.
%	\item modified Rayleigh number:
%	\begin{equation*}
%	\modRayleigh=\frac{\alpha gk D}{\nu\Omega}=\frac{\Rayleigh\Ekman}{\Prandtl}
%	\end{equation*}
%	\item Reynolds number:
%	\begin{equation*}
%		\Reynolds=\frac{u D}{\nu}=\tilde{u}_\mathrm{rms.}\,,
%	\end{equation*}
%	where $\tilde{u}_\mathrm{rms.}$ is the dimensionless rms-value of the velocity.
%	\item magnetic Reynolds number:
%	\begin{equation*}
%		\magReynolds=\frac{uD}{\eta}=\Reynolds\magPrandtl=\tilde{u}_\mathrm{rms.}\magPrandtl
%	\end{equation*}
%	\item Elsasser number:
%	\begin{equation*}
%		\Elsasser=\frac{B^2}{\rho\gmu_0\eta\Omega}=\tilde{B}_\mathrm{rms}^2\,,
%	\end{equation*}
%	where $\tilde{B}_\mathrm{rms}$ is the dimensionless rms-value of the magnetic field.
\end{itemize}
Here, $D$ denotes a reference length of the problem, which is typically the shell thickness for spherical problems. Furthermore, $\nu$ is the kinematic viscosity, $\kappa$ and $\eta$ are thermal and magnetic diffusivities, $\alpha$ is the thermal expansion coefficient, and $g$ is a reference gravity magnitude. The dimensionless gravity vector is the only possible field additional variable of the problem. Commonly a constant or linearly varying model in radial direction is chosen for spherical shell problems. The rotation vector is not a field but may be time-dependent. 

First, a discretization of the hydrodynamic case including Eqs.\,\eqref{eqn:Momentum} and \eqref{eqn:Energy} is considered. Then an extension to the full MHD system including the magnetic field~$\bm{B}$ is discussed. Finally the benchmark cases of \citeauthor{Christensen2001} are presented. 
\section{The hydrodynamic problem}
In the hydrodynamic case, the magnetic field is not present and the Lorentz force in Eq.\,\eqref{eqn:Momentum} vanishes. We introduce the following short-hand notations for volume and surface integrals
\begin{equation*}
	\inner{\bm A}{\bm B} = \int\limits_{\Omega} \bm A \star \bm B\, \d V \; , \quad
	\innerSurf{\bm A}{\bm B}  = \int\limits_{\Gamma} \bm A \star \bm B\, \d A \ ,
\end{equation*}
where $\bm A \star \bm B$ represents the contraction of two tensors $\bm A$ and $\bm B$ of arbitrary rank to a scalar. Furthermore, we introduce operators related to viscosity/diffusion~($\mathcal{A}$), incompressible/pressure~($\mathcal{B}$) and convection~($\mathcal{C}$).
\begin{align*}
\begin{aligned}
	\elliptic{\bm{\phi}}{\bm{\psi}}&=\inner{\nabla\bm{\phi}}{\nabla\bm{\psi}}\,, &
	\saddle{\bm{\phi}}{\psi}&=\inner{\nabla\cdot\bm{\phi}}{\psi}\,, &
	\convec{\bm{\phi}}{\bm{\psi}}{\bm{\chi}}&=\inner{\bm{\phi} \cdot (\nabla\bm{\psi})}{\bm{\chi}}\,.
\end{aligned}
\end{align*}
The test functions are denoted by $Q$, $\bm{w}$ and $U$. We test the incompressibility constraint by $Q$, the momentum equation by $\bm{w}$ and the energy equation by $U$. The problem considered herre shall be a Dirichlet problem for the velocity and the temperature. This has the consequence that the test functions vanishes on the boundary and the boundary terms that occur when applying integration by parts vanishes as well. The weak form of the problem reads: Find $\bm{v}$, $P$ and $T$ such that
\begin{align}
	\saddle{\bm{v}}{Q}&=0\,,\label{eqn:WeakIncompressibility} \\
	\inner{\pd{\bm{v}}{t}}{\bm{w}}+\convec{\bm{v}}{\bm{v}}{\bm{w}}+C_1\inner{\bm{\Omega}\times\bm{v}}{\bm{w}}&=\saddle{\bm{w}}{P}-C_2\elliptic{\bm{v}}{\bm{w}}+C_3\inner{T\bm{g}}{\bm{w}}\,,\label{eqn:WeakMomentum}\\
	\inner{\pd{T}{t}}{U}+\convec{\bm{v}}{T}{U}&=-C_4\elliptic{T}{U}\,,
	\label{eqn:WeakEnergy}
\end{align}
holds for all test functions $\bm{w}$, $Q$, $U$. Note that due to integration by parts the signs associated to $\mathcal{A}$ and $\mathcal{B}$ are reversed on the right-hand sides of Eqs.\,\eqref{eqn:WeakMomentum} and \eqref{eqn:WeakEnergy}.

\subsection{Time discretization}
The time stepping method is discussed first in context with the differential equation (strong form). An application to the weak is straightforward and presented afterwards. For the time stepping, almost all of the codes discussed in \cite{Matsui2016} use the Crank-Nicolson scheme for the linear terms and Adams-Bashforth extrapolation for the other terms. This scheme falls in the category of linear implicit-explicit multistep methods or IMEX schemes, see \cite{Ascher1995}. IMEX schemes treat the linear stiff terms implicitly and the non-linear terms are treated explicitly. The main advantage of these schemes in contrast to other stiffly accurate schemes is that the resulting system of equations is linear in each time step, which is computationally less expensive in contrast to, for example, diagonal implicit Runge-Kutta schemes (DIRK). DIRK schemes require a solution of a non-linear problem during each stage. In summary, DIRK methods allow adaptive time stepping through embedded error computation and may possess a larger domain of stability but they are computationally much more expensive. Recently, Runge-Kutta methods and IMEX schemes have combined and are sometimes referred to as additive Runge-Kutta schemes (ARK), see \cite{Ascher1997,Kennedy2003,Nikitin2005}. These schemes apply a splitting of the right-hand into implicit and explicit terms while preserving the properties of a Runge-Kutta scheme. The main advantage of these schemes is that only a linear system needs to be solved in each time step and an adaptive time step control is applicable if the Runge-Kutta scheme has an embedded lower order variant. Such schemes are made publicly available, for example, through the SUNDIALS library, see \cite{Hindmarsh2005}.

In the following, we focus on the classical second-order multistep IMEX schemes. In \cite{Ascher1995}, these are presented by considering the scalar convection-diffusion equation
\begin{equation}
	\pd{u}{t}=f(u)+g(u),\quad\text{with}\quad
	f(u)=-\bm{a}\cdot\nabla u\quad\text{and}\quad
	g(u)=\nu\laplace u\,,
\end{equation}
where $\bm{a}$ shall denote an convection field. The term~$g$ is treated implicitly and the term~$f$ is treated explicitly. If $u^n=u(t_n)$ and $k_n=t_{n+1}-t_n$, the resulting IMEX scheme reads:
\begin{equation}
	\frac{1}{k_n}\left(\alpha_1 u^{n+1}+ \alpha_2 u^{n}+\alpha_3 u^{n-1}\right)=\beta_2 f(u^n) + \beta_3 f(u^{n-1})+\gamma_1 g(u^{n+1}) +\gamma_2 g(u^n)+\gamma_3 g(u^{n-1})\,,
\end{equation}
where $\alpha_i$, $\beta_i$ and $\gamma_i$ are the coefficients of the time stepping scheme. Table\,\ref{tbl:TimeSteppingCoefficients} lists the coefficients of the common IMEX schemes found in literature for constant and adaptive or variable step size time stepping. If the step size is variable the time step ratio~$\omega_n$ enters the coefficients. It is defined by $\omega_n=k_n/k_{n-1}$. All schemes listed are second order in time and the main difference between is their stability region, which depends on the ration of the viscous term, $g$, relative to the convective term, $f$.
\begin{table}[!htb]
	\centering
	\caption{Coefficients of typical IMEX schemes, see \cite{Ascher1995,Wang2008}. CNAB---Crank-Nicolson-Adams-Bashforth, SBDF2---semi-implicit backward differentiation formula, MCNAB---modified Crank-Nicolson-Adams-Bashforth, CNLF---Crank-Nicolson leap frog.\label{tbl:TimeSteppingCoefficients}}
	\begin{tabular}{ccrrr|ccc}
		\toprule
		\multirow{3}{*}{type}& & \multicolumn{5}{c}{time stepping} \\
		& & \multicolumn{3}{c|}{constant} & \multicolumn{3}{c}{adaptive}\\
		& $i$ & 1 & 2 & 3 & 1 & 2 & 3 \\
		\midrule
		\multirow{3}{*}{CNAB} & $\alpha_i$ & $1$ & $-1$ & $0$ &
		$1$ & $-1$ & $0$\\
		& $\beta_i$ & -- & $\frac{3}{2}$ & $-\frac{1}{2}$ & 
		-- & $(1+\frac{1}{2}\omega_n)$ & $-\frac{1}{2}\omega_n$\\
		& $\gamma_i$ & $\frac{1}{2}$ & $\frac{1}{2}$ & $0$ &
		$\frac{1}{2}$ & $\frac{1}{2}$ & $0$\\
		\midrule
  		\multirow{3}{*}{SBDF2} & $\alpha_i$ & $\frac{3}{2}$ & $-2$ &
  		$\frac{1}{2}$ & $\frac{1+2\omega_n}{1+\omega_n}$ & $-(1+\omega_n)$ & $\frac{\omega_n^2}{1+\omega_n}$\\
  		& $\beta_i$ & -- & $2$ & $-1$ &
  		-- & $(1+\omega_n)$ & $-\omega_n$\\
  		& $\gamma_i$ & $1$ & $0$ & $0$ & $1$ & $0$ & $0$\\
  		\midrule
  		\multirow{3}{*}{MCNAB} & $\alpha_i$ & $1$ & $-1$ & $0$ & $1$ & $-1$ & $0$\\
  		& $\beta_i$  & -- & $\frac{3}{2}$ & $-\frac{1}{2}$ & 
		-- & $(1+\frac{1}{2}\omega_n)$ & $-\frac{1}{2}\omega_n$\\
  		& $\gamma_i$ & $\frac{9}{16}$ & $\frac{3}{8}$ & $\frac{1}{16}$ & 
  		$\frac{8\omega_n+1}{16\omega_n}$ & $\frac{7\omega_n-1}{16\omega_n}$ & $\frac{1}{16}$\\
  		\midrule
  		\multirow{3}{*}{CNLF} & $\alpha_i$ & $1$ & $-1$ & $0$ 
  		& $\frac{1}{1+\omega_n}$ & $\omega_n-1$ & $-\frac{\omega_n^2}{1+\omega_n}$\\
  		& $\beta_i$ & -- & $1$ & $0$ & -- & $1$ & $0$ \\
  		& $\gamma_i$  & $\frac{1}{2}$ & $\frac{1}{2}$ & $0$ &
  		$\frac{1}{2\omega_n}$ & $\frac{\omega_n-1}{2\omega_n}$ & $\frac{1}{2}$\\
  		\bottomrule
	\end{tabular}
\end{table}

If an IMEX schemes is applied to the hydrodynamic equations, the stiff term ($g$) in the momentum equation is the viscous term, $\laplace\bm{v}$, and in the energy equation it is the conduction term, $\laplace T$. The non-linear  term ($f$) is the convective term $\bm{v}\cdot\nabla\bm{v}$ and $\bm{v}\cdot\nabla T$, respectively. Up to now, I have treated the other terms in the momentum equation as source terms and used extrapolation from the previous two time steps. In order to ensure incompressibility the pressure is treated implicit. Finally, the resulting system of linear equations to compute the solution at the next time level, \ie, $\bm{v}^{n+1}$, $p^{n+1}$ and $T^{n+1}$, reads:
\begin{gather}
	\nabla\cdot\bm{v}^{n+1}=0\,,\\
	\begin{multlined}[c]
		\frac{1}{k_n}\left(\alpha_1 \bm{v}^{n+1}+ \alpha_2 \bm{v}^{n}+\alpha_3 \bm{v}^{n-1}\right)=-\left(\beta_2\bm{v}^{n}\cdot\nabla\bm{v}^{n}+\beta_3\bm{v}^{n-1}\cdot\nabla\bm{v}^{n-1}\right)-\nabla P^{n+1}+{}\\
		+C_2\left(\gamma_1\laplace\bm{v}^{n+1}+\gamma_2\laplace\bm{v}^{n}+\gamma_3\laplace\bm{v}^{n-1}\right)-C_1\bm{\Omega}\times\bm{v}^*-C_3T^*\bm{g}\,,
	\end{multlined}\label{eqn:MomentumIMEX}\\
	\begin{multlined}[c]
		\frac{1}{k_n}\left(\alpha_1 T^{n+1}+ \alpha_2 T^{n}+\alpha_3 T^{n-1}\right)=-\left(\beta_2\bm{v}^{n}\cdot\nabla T^{n}+\beta_3\bm{v}^{n-1}\cdot\nabla T^{n-1}\right)+{}\\
		+C_4\left(\gamma_1\laplace T^{n+1}+\gamma_2\laplace T^{n}+\gamma_3\laplace T^{n-1}\right)\,,
	\end{multlined}\label{eqn:EnergyIMEX}
\end{gather}
where $T^*$ and $\bm{v}^*$ are the extrapolated quantities and computed as
\begin{equation}
	\bm{v}^*=\left(1+\omega_n\right)\bm{v}^{n-1}-\omega_n\bm{v}^{n-2}\,,\qquad
	T^*=\left(1+\omega_n\right)T^{n-1}-\omega_nT^{n-2}\,.
\end{equation}
Reordering Eqs.\,\eqref{eqn:MomentumIMEX} and \eqref{eqn:EnergyIMEX} and using the abbreviations $\bm{a}_v=(\alpha_2 \bm{v}^{n}+\alpha_3 \bm{v}^{n-1})/k_n$ for the accelerative part on the right-hand side, $\bm{c}_v=\beta_2\bm{v}^{n}\cdot\nabla\bm{v}^{n}+\beta_3\bm{v}^{n-1}\cdot\nabla\bm{v}^{n-1}$, $\bm{d}_v=\gamma_2\laplace\bm{v}^{n}+\gamma_3\laplace\bm{v}^{n-1}$ and likewise $a_T$, $c_T$ and $d_T$ for part occurring in the energy equation gives:
\begin{gather}
	\frac{\alpha_1}{k_n}\bm{v}^{n+1}-C_2\gamma_1\laplace\bm{v}^{n+1}+\nabla P^{n+1}=-\bm{a}-\bm{c}_v+C_2\bm{d}_v-C_1\bm{\Omega}\times\bm{v}^*-C_3T^*\bm{g}\,,\label{eqn:MomentumIMEXReorder}\\
	\frac{\alpha_1}{k_n}T^{n+1}-C_4\laplace T^{n+1}=-a_T-c_T+C_4d_T\,.
	\label{eqn:EnergyIMEXReorder}
\end{gather}

\subsection{Linear system}
If Eqs.\,\eqref{eqn:MomentumIMEXReorder} and \eqref{eqn:EnergyIMEXReorder} are multiplied by the test functions in order to obtain the weak formulations, the following linear system results for the coefficient vectors $\mathbf{v}$, $\mathrm{p}$ and \gtheta for velocity, presure and temperature:
\begin{equation}
	M\begin{bmatrix} \mathbf{v} \\ \mathrm{p} \end{bmatrix}
	=\begin{bmatrix}
		\mathbf{A} & B\\
		B^T & 0
	\end{bmatrix}
	\begin{bmatrix}	\mathbf{v} \\ \mathrm{p}	\end{bmatrix}
	=\begin{bmatrix} \mathbf{f} \\ 0 \end{bmatrix}\quad\text{and}\quad
	A_T\gtheta=\mathrm{g}\,,
\end{equation}
where the matrices $\mathbf{A}$, $B$ and $A_T$ are defined as explained below. The matrix~$\mathrm{A}$ represents the discretization of the velocity part of the left-hand side of Eq.\,\eqref{eqn:MomentumIMEXReorder}. With $\bm{\phi}$ being a function from finite element space of the velocity, it is given by:
\begin{equation}
	\mathbf{A}=\frac{\alpha_1}{k_n}\mathbf{M}+C_2\gamma_1\mathbf{K}\,,\quad
	\text{with}\quad
	\mathbf{M}_{ij}=\int\bm{\phi}_i\cdot\bm{\phi}_j\drm{V}\quad
	\text{and}\quad
	\mathbf{K}_{ij}=\int\nabla\bm{\phi}_i\cdott\nabla\bm{\phi}_j\drm{V}\,.
\end{equation}
Clearly, $\mathbf{M}$ is the velocity mass matrix and~$\mathbf{K}$ the velocity stiffness matrix. The matrix $B$ results from the pressure term in the momentum equation. It is the discrete equivalent of the operator~$\mathcal{B}$ defined above and its components are computed by: $B_{ij}=\int(\nabla\cdot\bm{\phi}_i)\psi_i\drm{V}$, where $\psi$ is function form the finite element space of the pressure. Finally, the matrix~$A_T$ is the scalar equivalent of the matrix~$\mathbf{A}$. Hence,
\begin{equation}
	A_T=\frac{\alpha_1}{k_n}M+C_4\gamma_1K\,,\quad
	\text{with}\quad
	M_{ij}=\int\varphi_i\varphi_j\drm{V}\quad
	\text{and}\quad
	K_{ij}=\int\nabla\varphi_i\cdot\nabla\varphi_j\drm{V}\,.
\end{equation}
Note that the temperature equation is decoupled from the momentum equation. Hence, a possibility to circumvent the extrapolation of the buoyancy term in the momentum equation by $T^*$ is to first solve the temperature equation in each time step, then assemble the right-hand side with the correct buoyancy term including $T^{n+1}$ and then solve for the velocity.

\subsection{Linear solvers and Schur complement approximation}
The matrices~$\bm{A}$ and $A_T$ defined above have nice properties because both are symmetric and positive definite. For these class of matrices the most efficient iterative linear solver is a preconditioned conjugate gradient method (CG) with multigrid preconditioning. These methods allow to solve the linear system with number of linear iterations that is independent of the size of the system. The Trilinos and PETSc libraries provides highly efficient parallel multigrid preconditioners, which can be accesses through an interface of the deal.ii library \cite{deal.ii-9.0,Trilinos,PETSc}.

The Stokes system consisting of the block matrix $M$ cannot be solved with a such a simple method because although symmetric in case of an IMEX scheme it is not positive definite. It is easy to see that there is at least one zero eigenvalue because the pressure is only defined up to a constant. However an inverse exists if the matrix $\mathbf{A}$ is invertible, which is the case here. By block elimination we find that the matrix~$M$ has the following $L$-$U$-decomposition, see \cite{Elman2014},
\begin{equation}
	M
	=\begin{bmatrix} \mathbf{A} & B\\ B^T & 0 \end{bmatrix}
	=\begin{bmatrix} \mathbf{I} & 0\\ B\mathbf{A}^{-1} & I \end{bmatrix}
	\begin{bmatrix} \mathbf{A} & B^T\\ 0 & -S \end{bmatrix}\quad\text{with}\quad
	S=B\mathbf{A}^{-1}B^T\,,
	\label{eqn:StokesLU}
\end{equation}
where $S$ is the Schur complement. The matrix~$S$ is the magic ingredient here because it is not feasible to compute this matrix. Moreover the inverse of $\mathbf{A}$ would be a dense matrix and therefore $S$ would be dense as well. The idea is now to precondition the GMRES method with an approximation of the $L$-$U$-decomposition above. Specifically the Schur complement is replaced by an approximation, whose inverse is be computed easily. The construction of the correct Schur is the topic of numerous papers published in the last two decades. An good introduction related to the Navier-Stokes system is given in \cite{Elman2014} and a broader view is taken in the review paper of \citeauthor{Benzi2005}.

For the problem at hand, the suitable Schur complement approximation is given by the so-called Cahouet-Chabbard preconditioner, which also attributed to Bramble-Pasciak, see \cite{Benzi2005,Bramble1997,Cahouet1988,Mardal2004}. The approximation of $S$ is denoted by $\hat{S}$ and its inverse action is given by:
\begin{equation}
	\hat{S}^{-1}=\frac{\alpha_1}{k_n}K_p^{-1}+C_1\gamma_1M_p^{-1}\,,
\end{equation}
where $K_p$ is a stiffness matrix on the pressure space and $M_p$ is the mass matrix. Note the linear systems associated to $K_p$ and $M_p$ are symmetric positive definite, for which efficient iterative methods are available. However, the choice how to generate the stiffness matrix is crucial for the preconditioner to be successful. Note that there are no boundary conditions on the pressure (pure Neumann problem), which makes the stiffness matrix singular. A successful method is the approximate the stiffness matrix by $K_p=B \diag{\mathbf{M}}^{-1} B^T$, \cf \cite{Cahouet1988}. Assembling of $K_p$ explicitly and constraining a single degree of freedom on the pressure space has not been successful but may be more efficient.

In summary, the preconditioner~$P$ is given by an approximation of the upper triangular matrix in Eq.\,\eqref{eqn:StokesLU}.
\begin{equation}
	P=\begin{bmatrix} \hat{\mathbf{A}} & B^T \\ 0 & -\hat{S}\end{bmatrix}\qquad
	P^{-1}=\begin{bmatrix} \hat{\mathbf{A}}^{-1} & \hat{\mathbf{A}}^{-1}B^T\hat{S}^{-1} \\ 0 & -\hat{S}^{-1}\end{bmatrix}\,.
\end{equation}
In fact, only the application of the inverse of $P$ is required and $P$ itself is never constructed. From a first inspection, the inverse of $P$ involves a lot of solution of linear system but actually there are just three system involved. By block algebra, it is easy to see that the system may be solve by going through the following steps:
\begin{enumerate}
	\item Solve $-\hat{S}\mathrm{p}=\mathrm{g}$ for the pressure $\mathrm{p}$. This involves two linear solves. 
	\item Compute $\hat{\mathbf{f}}=\mathbf{f}-B^T\mathrm{p}$.
	\item Solve $\hat{\mathbf{A}}\mathbf{v}=\hat{\mathbf{f}}$ for the velocity $\mathbf{v}$. This operation is often replaced by an application of the preconditioner, which is computationally less cumbersome.
\end{enumerate}

A concluding remark is concerned with a comparison of the Schur complement approach to classical pressure projection. The review paper by \citeauthor{Guermond2006} points out that pressure projection separate the problem into a solution for the velocity and one for the pressure. This requires two linear solves per time step only. However, the pressure projection schemes presented in \cite{Guermond2006} rely on the application of backward differentiation formulas for the time derivative and a discussion of the treatment of the non-linear term is omitted. In how far the pressure projection schemes can be extended to higher order schemes is questionable but recent publication indicate that the IMEX Runge-Kutta schemes combined with a segregated treatment of the pressure may be a successful approach \cite{Ansorge2017,Colomes2015}.

\subsection{Preconditioner performance}
The results below show the performance of the preconditioner for the depending on the size of the mesh and the length of the time step. A start-up phase of the non-rotating hydrodynamic case is computed with $\Prandtl=1$ and $\Rayleigh=\num{e5}$.
\begin{table}
	\centering
	\caption{Dependency on the time step for the first 25 iterations.}
	\begin{tabular}{rc}
		\toprule
		\multicolumn{1}{c}{$k_n$} & $n_\mathrm{iter.}$\\
		\midrule
		\num{1e-1} & 23 to 24\\
		\num{1e-2} & 12 to 14\\
		\num{1e-3} & 11 to 12\\
		\num{1e-4} & 9  to 10\\
		\num{1e-5} & 6  to 7\\
		\bottomrule
	\end{tabular}
\end{table}

\begin{table}
	\centering
	\caption{Dependency on the time step for the first 25 iterations.}
	\begin{tabular}{rrc}
		\toprule
		\multicolumn{1}{c}{level} & \multicolumn{1}{c}{DoFs} & $n_\mathrm{iter.}$\\
		\midrule
		2 & \num{4608} & 9 to 10\\
		3 & \num{13920} & 9 to 10\\
		4 & \num{41664} & 9 to 10\\
		5 & \num{138624} & 8 to 9\\
		6 & \num{498432}  & 8 to 9\\
		7 & \num{1881600} & 8 to 9\\
		\bottomrule
	\end{tabular}
\end{table}

\section{The magnetic problem}
For the magnetic problem a vector potential formulation is used and the weak form of the induction is derived from Maxwell's equation after the magnetohydrodynamics approximation, \ie, $\order{\nicefrac{\p\bm{D}}{\p t}}\ll\order{\nabla\times\bm{H}}$. The system reads
\begin{equation}
	\nabla\cdot\bm{B}=0\,,\qquad
	\pd{\bm{B}}{t}+\nabla\times\bm{E}=0\,,\qquad
	-\nabla\times\bm{H}=\bm{J}
\end{equation}
and is supplemented by the \ae ther relation $\bm{H}=\bm{B}/\gmu_0$. A vector potential~$\tilde{\bm{A}}$ having the property $\bm{B}=\nabla\times\tilde{\bm{A}}$ is introduced. This yields
\begin{equation}
	\nabla\times\Big(\pd{\tilde{\bm{A}}}{t}+\bm{E}\Big)=\bm{0 \quad }\Rightarrow \quad 
	\bm{E}=-\pd{\tilde{\bm{A}}}{t}-\nabla\phi\,.
\end{equation}
We apply Ohm's law for a moving conductor, \ie, $\bm{J}=\sigma(\bm{E}+\bm{v}\times\bm{B})$, where the convective current $q\bm{v}$ is neglected due to the magnetohydrodynamic approximation. Then Amp\`ere's law reads:
\begin{equation}
	\frac{1}{\gmu_0\sigma}\nabla\times\bm{B}=(\bm{E}+\bm{v}\times\bm{B})\,.
\end{equation}
By introducing the magnetic diffusivity~$\eta=\nicefrac{1}{\gmu_0\sigma}$ and substituting the expression for the magnetic field and the electric field we have
\begin{equation}\label{eqn:InductionPotential}
	\nabla\times(\eta\nabla\times\tilde{\bm{A}})=\Big(-\pd{\tilde{\bm{A}}}{t}-\nabla\phi+\bm{v}\times(\nabla\times\tilde{\bm{A}})\Big)\,.
\end{equation}
In order to eliminate the scalar potential~$\phi$ from the equation above, a shift of the vector potential is performed, which is given by
\begin{equation}
	\pd{\bm{A}}{t}=\pd{\tilde{\bm{A}}}{t}+\nabla\phi \quad \Leftrightarrow \quad 
	\bm{A}=\tilde{\bm{A}}+\int\limits_0^t\nabla\phi\drm{\tau}\,,\quad
	\bm{A}(\bm{x},0)=\tilde{\bm{A}}(\bm{x},0)\,.
\end{equation}
Since $\nabla\times\tilde{\bm{A}}=\nabla\times\bm{A}$, Eq.\,\eqref{eqn:InductionPotential} can now be expressed as:
\begin{equation}
	\label{eqn:InductionPotentialMod}
	\pd{\bm{A}}{t}+\nabla\times(\eta\nabla\times\bm{A})=\bm{v}\times(\nabla\times\bm{A})\,.
\end{equation}
To obtain a weak form the double curl term is transformed using integration by parts.\footnote{The vector calculus identity applied here reads: $\nabla\cdot(\bm{u}\times\bm{v})=(\nabla\times\bm{u})\cdot\bm{v}-\bm{u}\cdot(\nabla\times\bm{v})$.} Let $\bm{A}'$ denote the test function, then the weak form of Eq.\,\eqref{eqn:InductionPotentialMod} can be expressed as: Find $\bm{A}\in H(\curl)$, such  that
\begin{equation}
	\left(\pd{\bm{A}}{t},\bm{A}'\right)_{\!\mathrlap{\Omega}\ }
	+\left(\eta\nabla\times\bm{A},\nabla\times\bm{A}'\right)_{\mathrlap{\Omega}\ \ }+\left\langle\bm{n}\times(\eta\nabla\times\bm{A}),\bm{A}'\right\rangle_{\p\Omega}=\left(\bm{v}\times(\nabla\times\bm{A}),\bm{A}'\right)_{\Omega}
\end{equation}
holds for all $\bm{A}'\in H(\curl, \Omega)$.

In the non-dimensional case, the magnetic diffusivity is replaced by the dimensionless nubmer~$C_5$ or magnetic Prandtl number, see Section\,\ref{sec:Introduction}. Hence, the weak form is given by: Find $\bm{A}\in H(\curl)$, such  that
\begin{equation}
	\label{eqn:WeakFormA}
	\left(\pd{\bm{A}}{t},\bm{A}'\right)_{\!\mathrlap{\Omega}\ }
	+C_5\left(\nabla\times\bm{A},\nabla\times\bm{A}'\right)_{\mathrlap{\Omega}\ \ }
	+C_5\left\langle\bm{n}\times(\nabla\times\bm{A}),\bm{A}'\right\rangle_{\p\Omega}=\left(\bm{v}\times(\nabla\times\bm{A}),\bm{A}'\right)_{\Omega}
\end{equation}
holds for all $\bm{A}'\in H(\curl, \Omega)$.

Clearly the magnetic field is continuous across interface if the surrounding domain is an isolator. The following transition conditions apply on the boundary~$\p \Omega$:
\begin{equation}\label{eqn:JumpCondition}
\begin{aligned}
	\bm{n}\cdot\jump{\bm{B}}&=0\,, &
	\bm{n}\times\jump{\bm{B}}&=\bm{0}\,, &
	\bm{x}&\in\p\Omega\,.
\end{aligned}
\end{equation}
There are multiple possibilities to handle boundary term in the weak form: 
\begin{enumerate}
	\item The pseudo-vacuum boundary $\bm{n}\cdot\bm{B}=0$ on the boundary~$\p\Omega$ could be applied. Then the boundary term in the weak form vanishes. However in this case, there is no boundary condition available to constrain $\bm{A}$ (pure Neumann problem).
	\item The domain~$\Omega$ could be extended by a large exterior domain. In the exterior domain the same equation but with $\bm{v}=\bm{0}$ and $C_5=1$ is solved to mimic the vacuum. On the boundary of the exterior boundary the magnetic potential is commonly set to zero then.
	\item In an isolating exterior domain, a scalar potential exists, which has an analytical solution for spherical geometries in form of a series. The coefficients of a truncated series could be coupled to the solution for the magnetic obtained from the weak form.
	\item The domain~$\Omega$ could be extended by a large exterior domain. On the exterior domain the Laplace equation is solved for the scalar potential. This reduces the number of degrees of freedom because scalar elements can be used. The weak form in Eq.\,\eqref{eqn:WeakFormA} is tailor-made coupled with the one of the Laplace problem. In order to couple both potentials, the transition conditions are utilized. We have:
	\begin{gather}
		\bm{n}\times\jump{\bm{B}}=\bm{0}\quad \Leftrightarrow \quad \bm{n}\times\bm{B}_\mathrm{int.}=\bm{n}\times\bm{B}_\mathrm{ext.} \quad \Leftrightarrow \quad \bm{n}\times(\nabla\times\bm{A})=\bm{n}\times\nabla\vp\,,
	\end{gather}
	where $\varphi$ is the scalar potential in the exterior domain. Applying integration by parts on the boundary, \cite[58\psq]{Monk2003}, the boundary term in Eq.\,\eqref{eqn:WeakFormA} can be recasted to:
	\begin{equation}
		\left\langle\bm{n}\times(\nabla\times\bm{A}),\bm{A}'\right\rangle_{\varGamma}
		=\left\langle\bm{n}\times\nabla\vp,\bm{A}'\right\rangle_{\varGamma}
		=\left\langle\vp,\bm{n}\cdot\nabla\times\bm{A}'\right\rangle_{\varGamma}\,.
	\end{equation}

	Next the weak form in the exterior domain $\Omega_\mathrm{ext.}$ is discussed. The boundary of the exterior is decomposed into a boundary~$\varGamma=\varOmega_\mathrm{int.}\cap\varOmega_\mathrm{ext.}$ that is shared with the exterior domain and a part~$\varGamma_\infty$, which represent exterior envelope of the domain. On the interface~$\varGamma$ the outward normals of both domains are opposed, \ie $\tilde{\bm{n}}=-\bm{n}$, where $\tilde{\bm{n}}$ is the outward normal of the domain~$\varOmega_\mathrm{ext.}$. With this decomposition of the domain the weak form of $\laplace\varphi=0$ in the exterior domain is given by: Find $\vp\in H^1(\Omega_\mathrm{ext.})$, such that
	\begin{equation}
		\label{eqn:WeakFormPhi}
		\left(\nabla\vp,\nabla\vp'\right)_{\mathrlap{\Omega_\mathrm{ext.}}\, \ \ \ }
		+\left\langle\bm{n}\cdot\nabla\vp,\vp'\right\rangle_{\varGamma}
		-\left\langle\tilde{\bm{n}}\cdot\nabla\vp,\vp'\right\rangle_{\varGamma_\infty}=0\,.
	\end{equation}
	for all $\vp'\in H^1(\Omega_\mathrm{ext.})$ gilt.
	
	Because of $\bm{n}\cdot\jump{\bm{B}}=0$ the boundary term above on $\varGamma$  can be replaced by the magnetic vector potential:
	\begin{equation}
		\label{eqn:RandPhi}
		\left\langle\bm{n}\cdot\nabla\vp,\vp'\right\rangle_{\varGamma}=\left\langle\bm{n}\cdot\nabla\times\bm{A},\vp'\right\rangle_{\varGamma}\,.
	\end{equation}
	This yields a symmetric coupling and with $\varphi=0$ on $\varGamma_\infty$ the coupled weak form is given by: Find $\vp\in H^1(\Omega_\mathrm{ext.})$ and $\bm{A}\in H(\curl, \Omega_\mathrm{int.})$, such that
	\begin{subequations}
		\label{eqn:WeakFormFinal}
		\begin{gather}
		\begin{multlined}[c]
		\left(\pd{\bm{A}}{t},\bm{A}'\right)_{\!\mathrlap{\Omega_\mathrm{int.}}\ }
		+C_5\left(\nabla\times\bm{A},\nabla\times\bm{A}'\right)_{\mathrlap{\Omega_\mathrm{int.}}\ \ }+C_5\left\langle\vp,\bm{n}\cdot\nabla\times\bm{A}'\right\rangle_{\varGamma}\\
		=\left(\bm{v}\times(\nabla\times\bm{A}),\bm{A}'\right)_{\Omega_\mathrm{int.}}\,,
		\end{multlined}\\
		\left(\nabla\vp,\nabla\vp'\right)_{\mathrlap{\Omega_\mathrm{ext.}}\,\ \ \ }+\left\langle\bm{n}\cdot\nabla\times\bm{A},\vp'\right\rangle_{\varGamma}-\left\langle\tilde{\bm{n}}\cdot\nabla\vp,\vp'\right\rangle_{\varGamma_\infty}=0\,.
		\end{gather}
	\end{subequations}
	holds for all $\vp'\in H^1(\Omega_\mathrm{ext.})$ and $\bm{A}'\in H(\curl, \Omega_\mathrm{int.})$.
	\item The fact that there is a Green's function of the Laplace equation can also be utilized in relation to the problem at hand. Boundary element methods make use of this fact and only require a discretization of the boundary. A mathematical treatment and analysis of the BEM-FEM coupling for the induction equation is discussed in \cite{Lemster2014,Lemster2010}. The BEM method is not a local method because all degrees of freedom are coupled on the boundary. Therefore the matrices are not sparse any more and this can become a problem when solving the linear system. However, there are special matrix types for boundary element matrices known as hierarchical matrices, which cluster the matrix entries into smaller classes. These allow to efficiently handle the data structures.
\end{enumerate}

\section{Benchmark cases of \citeauthor{Christensen2001}}
The benchmark cases of \citeauthor{Christensen2001} are specified in Table~\ref{tbl:BenchmarkParameters}. In Table~\ref{tbl:BenchmarkParameters} the hydro case refers to the solutions of Eqs.\,\eqref{eqn:Momentum} and \eqref{eqn:Energy} and the MHD case solve all three equations. Furthermore, the maximum spherical harmonics degree~$\ell_\mathrm{max.}$ and the number of radial points are specified. 
\begin{table}[!htb]
	\centering
	\caption{Benchmark parameters according to \cite{Christensen2001}.\label{tbl:BenchmarkParameters}}
	\begin{tabular}{ccc|ccccc}
		\toprule
		& \multicolumn{2}{c|}{discretization} & \multicolumn{5}{c}{model}\\
		case & $\ell_\text{max.}$ & $N_r$ & $\Rayleigh$ & $\Prandtl$ & $\magPrandtl$ & $\Ekman$ & $\modRayleigh$ \\\midrule
		hydro & 65  & 85 & \num{e5} & 1 & -- & \num{e-3} & \num{100}\\
		MHD & 65 & 85 & \num{e5} & 1 & 3 & \num{e-3} & \num{100}\\
		\bottomrule
	\end{tabular}
\end{table}
The inner boundary is denoted by $\Gamma_\mathrm{icb}$ and the outer boundary by $\Gamma_\mathrm{cmb}$. Then, the boundary conditions for the velocity and the temperature are given by:
\begin{equation}
	\bm{v}|_{\Gamma_\mathrm{icb}}=\bm{0}\,,\quad
	\bm{v}|_{\Gamma_\mathrm{cmb}}=\bm{0}\,,\quad
	T|_{\Gamma_\mathrm{icb}}=1\,,\quad
	T|_{\Gamma_\mathrm{cmb}}=0\,.
\end{equation}
The interior and exterior domain are modeled as an electric insulator and the magnetic field is adjusted to scalar potentials at inner and outer boundary.

The initial condition for the temperature is given by:
\begin{equation}
	T(\bm{x},t=\num{0})=\frac{a}{1-a}\left(\frac{1}{r}-1\right)+\frac{21}{\sqrt{17920\gpi}}(1-3\xi^2+3\xi^4-\xi^6)\sin^4(\theta)\cos(4\varphi)\,,
\end{equation}
where $\xi=(2r-(1+a))/(1-a)$, and $r$, $\theta$ and $\varphi$ are spherical coordinates. The initial condition for the velocity is a vanishing velocity field:
\begin{equation}
	\bm{v}(\bm{x},t=\num{0})=\bm{0}\,.
\end{equation}
In the MHD case the same initial conditions for the velocity and the temperature apply and the initial condition for the magnetic field is given by:
\begin{equation}
\begin{gathered}
	B_r=\frac{5}{8}\frac{1}{1-a}\left(1-6r-2\frac{a^4}{r^3}\right)\cos(\theta)\,,\quad
	B_\theta=\frac{5}{8}\frac{1}{1-a}\left(9r-8-\frac{a^4}{r^3}\right)\sin(\theta)\,,\\
	B_\varphi=5\sin\left(\gpi\frac{r-a}{1-a}\right)\sin(2\theta)\,.
\end{gathered}
\end{equation}
The boundary condition for the magnetic field is a continuous transition to two potential fields at the inner core and core mantle boundary:
\begin{equation}
	\bm{B}|_{\Gamma_\mathrm{icb}}=\nabla V_\mathrm{ic}\,,\qquad
	\bm{B}|_{\Gamma_\mathrm{cmb}}=\nabla V_\mathrm{m}\,.
\end{equation}

\section{Benchmark case of \citeauthor{Jackson2014}}
The boundary conditions for the velocity and the temperature are given by:
\begin{equation}
	\bm{v}|_{\Gamma_\mathrm{icb}}=\bm{0}\,,\quad
	\bm{v}|_{\Gamma_\mathrm{cmb}}=\bm{0}\,,\quad
	T|_{\Gamma_\mathrm{icb}}=1\,,\quad
	T|_{\Gamma_\mathrm{cmb}}=0\,.
\end{equation}
The interior and exterior domain are modeled as an electric insulator and the pseudo-vacuum condition is used as a boundary condition.

The initial condition for the temperature is given by:
\begin{equation}
T(\bm{x},t=\num{0})=\frac{a}{1-a}\left(\frac{1}{r}-1\right)+\frac{21}{\sqrt{17920\gpi}}(1-3\xi^2+3\xi^4-\xi^6)\sin^4(\theta)\cos(4\varphi)\,,
\end{equation}
where $\xi=(2r-(1+a))/(1-a)$, and $r$, $\theta$ and $\varphi$ are spherical coordinates. The initial condition for the velocity is a vanishing velocity field:
\begin{equation}
\bm{v}(\bm{x},t=\num{0})=\bm{0}\,.
\end{equation}
In the MHD case, the same initial conditions for the velocity and the temperature apply and the initial condition for the magnetic field is given by:
\begin{equation}
\begin{gathered}
	B_r=\frac{5}{8\sqrt{2}}\frac{-48a+(4+7a)6r-4(7+3a)r^2+9r^3}{r}\cos(\theta)\,,\\
	B_\theta=-\frac{15}{4\sqrt{2}}\frac{(r-a)(r-1)(3r-4)}{r}\sin(\theta)\,,\quad
	B_\varphi=\frac{15}{8\sqrt{2}}\sin\left(\gpi\frac{r-a}{1-a}\right)\sin(2\theta)\,.
\end{gathered}
\end{equation}

\section{New magnetic problem}
